Number theory is the most charming subject in mathematics.
It deals problems about integers, which computer is good at.
There are many problems on elementary number theory in the ICPC contest,
each can be solved by beautiful solution with elegant coding skill.

\section{Prime and Divisibility}
Let $m$ and $n$ be integers, if exist $k\in \mathbb{Z}$ such that $m=nk$,
we say $m$ is a multiple of $n$ and $n$ is a divisor, or factor of $m$, denoted $n\mid m$.

For integers $m, n(n\neq 0)$, there is an unique pair of $r, q \in \mathbb{Z}$,
such that $m=nq+r, 0 \leq r < \abs{n}$, in the case $r\neq 0$,
it is said to be the remindar of $m$ divided by $n$.
In C/C++ language, we use \verb|m % n| to calculate the remindar of $m$ divided by $n$,
but when $m$ is negative, the result $r$ will be negative if it's now zero, and $-\abs{n} < r < 0$.
To make it positive, we should code as \verb|(m % n + abs(n)) % n|,
or usually \verb|(m % n + n) % n| if provided $n$ is positive.

A prime is a natural number that has no positive divisors other than $1$ and itself.
other numbers are composite, except $1$, which is neither prime nor composite.
The first primes are $2, 3, 5, 7, 11, \dots$. There are $25$ primes before $100$, and $168$ primes before $1000$.
There are infinite many primes and the number of primes no greater than $n$ is approximately $n / \log n$.

\subsection{Primality Test}
For $n > 1$, if it can not be divided by any numbers between $1$ and $n$, it must be a prime.
But observe if $n$ can be divide by $d(1 < d < n)$, it also can be dividey by $n / d$,
either $d$ or $n / d$ must be smaller or equal to $\sqrt{n}$.
So it's enough to check all numbers from $2$ to $\left[\sqrt{n}\right]$.
The worse case is when $n$ is a prime or equals the square of a prime,
but for random cases, this algoritm will soon end with a quite small divisor of $n$.
In coding, we can use \verb|i * i <= n| to avoid computing the square root of $n$.
\lstinputlisting[language=c++]{./source/number/primality.cpp}

There are faster algorithms to do the primality test, such as Miller-Rabin test.
Most of these algorithm are complicate, we don't disscuss them in this book.
But we shall introduce the Eratosthenes's sieve, which generate a table of primes,
and can be used when there is frequent primality tests of small numbers.

\subsection{Eratosthenes's Sieve}
For an integer $n > 1$, if we want to figure out all primes numbers from $1$ to $n$,
any algorithm that test each number will take a long time.

The Eratosthenes's sieve is a very old algorithm dates back to around 200 BC,
it's idea is quite simple: for each numbers from $2$ to $n$, if it is a prime,
then mark all it's multiples other than itself to be composite,
then find the next number which has not been marked yet, it must be a prime.
Repeat the operation till the end. In fact, everytime we find a prime $p$,
we only have to mark $p^2, p(p + 1), p(p + 2), \dots p\left[{n \over p}\right]$.
\lstinputlisting[language=c++]{./source/number/eratosthenes.cpp}


\subsection{Factoring Numbers}
An important fact which will be frequently used is:
given an integer greater than $1$, it can be decomposite into product of primes,
and the decomposition is unique if we don't distinguish the order.
This is know as the foundament theorem of arithmetic.
In concrete term, for integer $n > 1$, we have
$$n = p_1^{e_1} \dots p_k^{e_k}$$
where $p_i$ are disditct primes and $e_i > 0$, this is called the standard factorization of $n$.

The solutions of many problems depend on the standard factorization of $n$.
If we already know $p$ is a prime factor of $n$,
we can repeat divide $n$ by $p$ many times until the qoutient can not be dividey by $p$ any more.
Suppose we divide $n$ by $p$ for $k$ times, finally the quotient $m$ can not divided by $p$,
then $n = p^k m$, now we only have to factor $m$, which is smaller than $n$.
So the problem of factoring number reduce to find a prime factor.
Obviously this problem can not be easier than judge a number is prime.
If we want to factor all numbers from $1$ to $n$, inspired by the Eratosthenes's sieve,
we can store a prime factor for each number during the process.


\section{Linear Equation}
\subsection{$ax + by = c$}
The equation $ax + by = c$, where $a, b, c$ are integers served as a bridge links several aspects of number theory.

\begin{thm}
The equation $ax + by = c$ has a solution if and only if $\gcd(a, b) \mid c$.
\end{thm}
To prove this theorem, we need this lemma
\begin{lem}
The set $\{ax + by \mid x, y \in \mathbb{Z}\}$ equals to $\{kd \mid k \in \mathbb{Z}\}$ for some $d$.
\end{lem}
\begin{prf}
Choose $d \in \{ax + by \mid x, y \in \mathbb{Z}\} \backslash \{0\}$ with smallest absolute value,
we assert $\{ax + by \mid x, y \in \mathbb{Z}\} = \{kd \mid k \in \mathbb{Z}\}$.
Let $d = ax_1 + by_1$, for $c = ax_2 + by_2$, there is $q, r$ such that $c = dq + r, (0 \leq r < \abs{d})$,
then $r = a(x_2 - x_1q) + b(y_2 - y_1q)$.
\end{prf}

Since $a, b \in \{ax + by \mid x, y \in \mathbb{Z}\}$, according to the lemma, $d \mid a$ and $d \mid b$,
therefore $d \mid \gcd(a, b)$. On the other hand, for every common divisor $e$ of $a$ and $b$,
$e \mid d$ because $d = ax_1 + by_1$. So $d = \gcd(a, b)$. $\qed$
Now we come to the conclusion that $\{ax + by \mid x, y \in \mathbb{Z}\}$ is the set of all multiples of $\gcd(a, b)$.
This conclusion only guarantees the existence of solutions for $ax + by = d$, but we still don't know how to find one.
So it's necessary to develop an algorithm to solve this problem.

\subsubsection{Extended Euclid's Algorithm}
Give $a, b$, the extended Euclid's algorithm finds a pair of $(x, y)$ such that $ax + by = \gcd(a, b)$.
Assume $a = bq + r$, if $(x_1, y_1)$ is a solution of $ax + by = d$,
then $(bq + r)x_1 + by_1 = d$, that is $b(y_1 + qx_1) + rx_1 = d$,
therefore $(y_1 + qx_1, x_1)$ is a solution of $bx + ry = d$.
If we already got a solution of $bx + ry = d$, which is $(x_2, y_2)$,
let $(y_1 + qx_1, x_1) = (x_2, y_2)$, we can figure out $(x_1, y_1)$.
\lstinputlisting{./source/number/euclid.cpp}

\subsubsection{Euclid's algorithm}
If we only wish to calculate the greatest common divisor of $a$ and $b$,
the above algorithm reduce to the classic Euclid's algorithm.
\lstinputlisting[language=c++]{./source/number/gcd.cpp}


\subsection{The Chinese Remainder Theorem}
\begin{thm}
If $(m_i,m_j)=1$ for all $i\neq j$, the linear congruence equations
$$\begin{cases}
x\equiv a_1\pmod{m_1}\cr
x\equiv a_2\pmod{m_2}\cr
\quad \vdots\cr
x\equiv a_n\pmod{m_n}\cr
\end{cases}$$
has unique solution modulo $m_1m_2\hdots m_n$.
\end{thm}
The solution can be constructive.
In fact $x=\sum a_iM_iM_i^\prime$ is the solution,
where $M_i={m_1m_2\cdots m_n\over m_i}$, $M_i^\prime$ is the inverse of $M_i$ modulo $m_1m_2\hdots m_n$.

If $O(m_1+m_2+\hdots+m_n)$ is bearable
(in fact this is the most common case because $m_1m_2\hdots m_n$ can't be too large),
an enumerative algorithm can be carry out:

Let $x=0$ initially, increase $x$ by $1$ each time until we got $x\equiv a_1\pmod{m_1}$;
then increase $x$ by $m_1$ each time until we got $x\equiv a_2\pmod{m_2}$;
generally, in $k$th step, increase $x$ by $m_1\cdots m_{k-1}$ each time until we got $x\equiv a_k\pmod{m_k}$.
Then we will got the solution after $n$ steps.

\section{Modular Arithmetic}
Given an integer $n$, we say $a$ congruence to $b$ module $n$ if $n$
divides $a - b$ exactly, or $n \mid (a - b)$, and write $a \equiv b \pmod n$.
The goodness of modular arithmetic is the number won't get very large.

\subsection{The Repeated Squaring Method}
To calculate $a^n \mod m$.
\lstinputlisting[language=c++]{./source/number/mod_exp.cpp}

\subsection{$ax = b \pmod n$}
This equation can be transformed to $n \mid (ax - b)$,
so there is an integer $y$, for which $ax - b = ny$, or equally $ax - ny = b$.
So it has a solution if and only if $\gcd(a, n) \mid b$.

In particular, when $\gcd(a, n) = 1$ and $b = 1$, there is a number $a^\prime$
for which $aa^\prime \equiv 1 \pmod n$, we call it the inverse of $a$ module $n$.

\begin{thm}[Fermat-Euler]
Let $\gcd(a, n) = 1$, then $a^{\varphi(n)} \equiv 1 \pmod n$.
\end{thm}
Fermat's little theorem is the special case when $n = p$, which is a prime.
\subsection{Index and Primitive root}
$\gcd(a, n) = 1$, the smallest $d$ such that $a^d \equiv 1 \pmod n$ is called the
index of $a$ module $n$.

\begin{thm}
The index of $a$ module $n$ is a divisor of $\varphi(n)$.
\end{thm}
According to this theorem, if we want to calculate the index of $a$ module $n$,
we can enumerate all divisors of $\varphi(n)$, but we have more efficient ways.

First decomposite $\varphi(n)$ into $p_1^{e_1} \dots p_k^{e_k}$.

\section{Carry System}

\section{Arithmetic Functions}
Generally speacking, an arithmetic function is a function definded on integers only.
And in ICPC, we only intrested in the arithmetic functions which take values as
integers too.

The common ones are $\phi(n), d(n), \sigma(n)$.

\subsection{M\"obius Transform}
\subsection{Dirichlet Convolution}

\section{Algebraic Methods}
We introduce some algebraic approaches to make elementary number theory simpler.
Beginners may skip this section.

\subsection{Finite Abelian Group}
Let $n > 1$ be an integer,
the complete residue classes module $n$ form a cyclic group $\mathbb{Z} / n \mathbb{Z}$,
which is isomorphic to $C_n$,
and the reduced residue classes module $n$ form a
finite Abelian group $(\mathbb{Z} / n \mathbb{Z})^\ast$.

When $n=2^k$, we have
\begin{eqnarray*}
(\mathbb{Z} / 2 \mathbb{Z})^\ast & \cong & \{ e \}	\cr
(\mathbb{Z} / 4 \mathbb{Z})^\ast & \cong &  C_2	\cr
(\mathbb{Z} / 8 \mathbb{Z})^\ast & \cong &  C_2 \oplus C_2	\cr
(\mathbb{Z} / 16 \mathbb{Z})^\ast & \cong &  C_2 \oplus C_4	\cr
(\mathbb{Z} / 32 \mathbb{Z})^\ast & \cong &  C_2 \oplus C_8	\cr
					& \vdots &	\cr
(\mathbb{Z} / 2^k \mathbb{Z})^\ast & \cong & C_2 \oplus C_{k-2}	\cr
\end{eqnarray*}
When $n=p^k$, where $p$ is an odd prime, we know $n$ do
have primitive roots according to elementary number theory,
thus $\mathbb{Z} / n \mathbb{Z}$ is a cyclic group
of order $\varphi(p^k) = p^k - p^{k-1}$.

% \subsection{Lattice}
