\section{Linear Algebra}
\subsection{Determinant}
Evulate a matrix.

$$
$$
\subsection{Linear Equations}


\section{Discrete Fourier Transform}

Let $P$ and $G$ be two polynomials of degree $n$, if we want to calculate their product, 
it will take $O(n^2)$.

Let $\{a_k\}_{k=0}^{n-1}$ be a sequence of $n$ numbers,
the discrete Fourier transform of it is another sequence $\{x_k\}_{k=0}^{n-1}$ defined by 
$$x_k = \sum_{r=0}^{n-1} a_r \exp({2\pi i kr \over n})$$
and the inverse transform is given by 
$$a_k = {1 \over n} \sum_{r=0}^{n-1} x_r \exp(-{2\pi i kr \over n})$$

Let $\{a_k\}$ and $\{b_k\}$ be two sequences of length $n$, 
the circular colvolution of them is anther sequence $\{c_k\}$ defined by
$$c_k = \sum_{r+s \equiv k \pmod n} a_r b_s$$

\subsection{Circular Convolution}

The discrete Fourier transform has something to to with circular convolution, 
that is $$a \circ b = \mathcal{F}^{-1}(\mathcal{F}(a) \cdot \mathcal{F}(b))$$

If we can calculate $\mathcal{F}(a)$ and $\mathcal{F}^{-1}(a)$ in $O(n \log n)$ time,
then multiply polynomials will be also in $O(n^2)$ time.

\subsection{Fast Fourier Transform}
To do fast Fourier transform, $n$ must be a power of $2$, 
if $n$ is not a power of $2$, we can choose an integer wich is a power of $2$
and greater than $n$, then add some padding zeros at the end of the sequence.

Let $w_N = e^{-{2\pi i \over N}}$, a primitive $N$-th unit root.
When $N$ is even, $w_{N / 2} = w_N^2$.
Let $x=(x_n)$ be a sequence of length $N$, 
$p_0, p_1, \dots, p_{N/2}$ be the DFT of $x_0, x_2, \dots, x_{N-2}$,
$q_0, q_1, \dots, q_{N/2}$ be the DFT of $x_1, x_3, \dots, x_{N-1}$,
$y$ be the DFT of $x$.
Then we have
$$p_n = \sum_{k=0}^{N/2 - 1} x_{2k} w_{N/2}^{nk} \quad n = 0, 1, \dots, N/2 - 1$$
$$q_n = \sum_{k=0}^{N/2 - 1} x_{2k+1} w_{N/2}^{nk} \quad n = 0, 1, \dots, N/2 - 1$$
and 
\begin{align*}
y_n &= \sum_{k=0}^{N-1} x_k w_N^{nk} \quad n = 0, 1, \dots, N - 1\\
&= \sum_{k=0}^{N/2-1} x_{2k} w_N^{2nk} + \sum_{k=0}^{N/2 - 1} x_{2k+1} w_N^{n(2k+1)} \\
&= \sum_{k=0}^{N/2-1} x_{2k} w_{N/2}^{nk} + w_N^n \sum_{k=0}^{N/2 - 1} x_{2k+1} w_{N/2}^{nk}
\end{align*}
therefore, when $0 \leq n < N / 2$, 
$$y_n = p_n + w_N^n q_n$$
and because of $w_{N/2}^{nk} = w_{N/2}^{(n+N/2)k}$, $w_N^{N/2} = -1$,
$$y_{n + N} = p_n + w_N^{n + N/2} q_n = p_n - w_N^n q_n.$$
According to this relation, 
if $p$ and $q$ are known, we can calculate $y$ only in $O(N)$ time,
but this requires $N$ to be even. If $N=2^l$, $p$ and $q$ also can be calculated
in the same way. So we obtained a recursive algorithm to calculate DFT, 
using only $O(N\log N)$ time, as long as $N = 2^l$.
The time complexity can be proved by induction,
when $N = 1$, we only let $y_0 = x_0$;
when $N = 2^l$, we can calculate both $p$ and $q$ in 
$$O({N\over 2}\log {N\over 2})$$ time,
so the total time is 
$$O({N\over 2}\log {N\over 2}) + O({N\over 2}\log {N\over 2}) + O(N) = O(N\log N)$$

Here gives the main process of FFT.
\verbatiminput{fft.txt}

\subsection{Number Theoretic Transform}
Instead of $\mathbb{C}$, we can perform DFT over a finite field, 
an $N$-th primitive root will be used in place of $w_N$.
It's most convenient to take the finte field $\mathbb{Z} / p\mathbb{Z}$, 
where p is a prime.
If $p=2^lk + 1$, then $\mathbb{Z} / p\mathbb{Z}$ contains
$2^l$-th primitive roots. According to Dirichlet's theorem on arithmetic progress,
given $l$ fixed, there are infinite many primes such that $p-1$ divides $2^l$.
In the range of $32$-bit integers, $2013265921 = 2^{27} \ast 15 + 1$ is a good choice.
