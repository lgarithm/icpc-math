\section{Permutation}
A \idx{permutation} is an arrangement of $n$ distinct objects in a certain order.
Usually the $n$ objects will be represented by integer numbers $1, \dots, n$.

\subsection{Group of Permutations}
A $n$-permutation $\varphi$ can be considered as a bijection
$$\varphi : \{1, \dots, n\} \to \{1, \dots, n\}$$
Therefore two permutations can be composited as functions.
Bijection has inverse. This makes the set of all permutations into a finite group,
called the \idx{symmetric group} $S_n$.

\subsection{Number of Permutations and Set Partitions of a Given Shape}
We say a permutation is of shape
$$1^{\lambda 1}2^{\lambda 2} \dots n^{\lambda n} \label{shape}$$
if it has exactly $\lambda_i$ cycles of length $i$.
And we also say a partition of a set is of shape~\ref{shape}
if it has exactly $\lambda_i$ subsets of cardinate $i$.
The number of permutations and set partitions of shape $(1)$
are well know as
$$n! \over \lambda_1! \lambda_2! \dots \lambda_n!
		1^{\lambda_1} 2^{\lambda_2} \dots n^{\lambda_n}$$
and
$$n! \over \lambda_1! \lambda_2! \dots \lambda_n!
		(1!)^{\lambda_1} (2!)^{\lambda_2} \dots (n!)^{\lambda_n}$$
respectively.

\section{Counting}
Enumerative combinatorics counts the cardinate of a finite set.
Let $X$, $A$ and $B$ be finite sets, then $\vert A\times B\vert=\vert A\vert \times \vert B\vert$.
And $\vert B^{A}\vert$, the number of functions $A\to B$ is $\vert B\vert^{\vert A\vert}$,
because $\forall a\in A$, $f(a)$ has $\vert B\vert$ choices, and the choices are independent.
The special case is $\{0,1\}^{X}=2^{\vert X\vert}$. Recall $\vert \mathcal{P}(X)\vert=2^{\vert X\vert}$
because every element must be in or not in a specific subset.
In fact, there is a one-to-one corresponding between $\mathcal{P}(X)$ and $\{0,1\}^{X}$
\footnote{that's why we also write $2^{X}$ for $\mathcal{P}(X)$.},
$A \to \chi_A$, $\chi_A(x)=\begin{cases}1, x\in A\cr 0, x\notin A\end{cases}$.
Since $\chi_A$ determines $A$ uniquely, we call it the character function of $A$.

\subsection{Elementary Counting}
\subsection{Inclusion-Exclusion Principle}
Let $A_n$ be a finite family of subsets of $A$,
define $C(x)=\sum_{i=1}^{n}\chi_{A_n}(x)$,
we have sieve formula:
$\#\{x\mid C(x)=0\}=\#A - \sum\#A_i + \sum\#(A_i\cap A_j) + \dots
 + (-1)^r\sum\#(A_{i_1}\cap\dots\cap A_{i_r}) + \dots
 + (-1)^n\sum\#(A_{i_1}\cap\dots\cap A_{i_n})$.
More generally, we have:
$\#\{x\mid C(x)=k\}=\sum_{r=k}^{n}{r\choose k}(-1)^{r+k}\sum\#(A_{i_1}\cap \dots \cap A_{i_r})$.
The proof is based on the equation $\sum_{r=k}^{m}{r\choose k}{m\choose r}(-1)^r=\delta(m,k)$.

\subsection{P\'olya Enumeration Theorem}
Let group $G$ acts on set $X$, then $$\#O(x)=[G:G_x].$$
Let $N$ be the number of oribits, we have
$$N=\sum{1\over \left| O(x)\right|}
={1\over{\left| G\right|}}\sum{\left| G_x\right|}
={1\over{\left| G\right|}}\sum fix(g).$$
That is, the number of orbits equals the average number of fixed points.

\subsection{Counting Geometric Shapes}
\subsubsection{Euler's formula $V-E+F=2$}
Euler's formula for planar graph is a very beautiful formula,
we shall use it to a class of counting problems.
There are three variables in the equation, only we know two of them, the third can be calculated.
\footnote{sample problem: Beijing 2004 Preliminary, Fourier's Lines}

A basic example is to calculate how many parts can $n$ lines cut the plane if no two lines parallel
to eath other and no three lines intersect at a point. To calculate it, we must translate it into a
finite planar graph by adding a very large circle containing all intersection points first.
Since there are $V={n\choose 2}+n={n(n-1)\over 2} + n$ vertices
and $E=n^{2} + n$ edges, $F=E-V+2={n(n+1)\over 2} + 2$, but we only count the number of faces inside the
circle, so the answer is ${n(n+1)\over 2} + 1$.

\subsubsection{Pick's theorem}
Another class of counting problems is related to two dimensional lattice.
Say counting the number of points whose coordinates are integers inside a shape.
To do this, we need a lemma. If the coordinates of a triangle's three vertices are integers,
and there is no other point inside or on the edges of that triangle whose coordinates are integers,
then the area of the triangle is ${1\over 2}$. We can create a planar graph by connect as many as possible
points whose coordinateds are integers inside or en edges of that shage,
and no two segments intersec properly, then the number of small triangles is equal to 2 times the area of that shape.
Then we calculate the total degree of all small triangles in two ways.
Every inside point contributes $2\pi$ degree and every point on edges except vertices contributes $\pi$,
the total degree contribute by vertices is $(n-2)\pi$.
let $i$ be the number of inside points,
$j$ be the number of points on boundary except vertices,
$n$ be the number of vertices,
$A$ be the area, we have $2A\pi = 2i\pi + j\pi + (n-2)\pi$,
that is $A=i+{j+n\over 2}-1$, or $A=i+{b\over 2}-1$,
$b$ is the number of points on doundary.

\section{Generating Function}
\subsection{Formal Power Series}
Generating function is a power tool to describe sequence. If $\{a_n\}$ is a sequence,
$\sum a_n x^n$ is its generating function. Since a sequence can be infinite, the generating
function is not a polynomial, but a formal series(by formal means we don't consider the convergence).
The sum of two formal series is defined by pointwise sum of their coefficients,
i.e. $\sum a_n + \sum b_n = \sum (a_n+b_n)$.
The product of two formal series is defined by the convolution of their coefficients,
i.e. $(\sum a_n)(\sum b_n) = \sum (\sum_{k=0}^{n} a_k b_{n-k})$.

Some commom generating functions are:
$${1\over{1-x}} = 1 + x + x^2 + \ldots + x^n + \ldots$$
$${1\over{(1-x)^{1+n}}} = {n\choose n} + {n+1\choose n}x + {n+2\choose n}x^2 + \ldots + {n + k\choose n}x^k + \ldots$$

Another kind of generating function of $\{a_n\}$ is $\sum {a_n\over n!} x^n$.

\subsection{Close Form}


\section{Combinatorics Numbers}
\subsubsection{binomial coefficient}
The binomial coefficients, $n\choose k$ are coefficients of $x$ in the expansion $(1+x)^n$.
More precisely, $(1+n)^n=\sum_{k=0}^n {n\choose k}x^k$.

\subsubsection{Catlan number}
Many enumeration problem has answer $C_n$,
where $C_n$ has the induction formula
$$C_{n+1} = \sum_{k=0}^n C_k C_{n-k}, C_0=1.$$
The first several Catalan numbers are:
$C_0=1$, $C_1=1$, $C_2=2$, $C_3=5$, $C_4=14$, $C_5=42$.
$$C_n={1 \over {n + 1}} {2n \choose n}$$

\subsubsection{partition number}
Let $p(n)$ denote the number of ways to write
$n$ into sums of positive numbers, regardless
of order, and $p(n,k)$ denote the number of ways
when there are exactly $k$ positive numbers,
we make the convention $p(0,0)=1$ and $p(n,k)=0$
when $n < k$ for convenience. The partition of
the first five positive numbers are listed below:
\newline
$1$	\newline
$2 = 1 + 1$	\newline
$3 = 2 + 1 = 1 + 1 + 1$	\newline
$4 = 3 + 1 = 2 + 2 = 2 + 1 + 1 = 1 + 1 + 1 + 1$	\newline
$5 = 4 + 1 = 3 + 2 = 3 + 1 + 1 = 2 + 2 + 1 = 2 + 1 + 1 + 1 = 1 + 1 + 1 + 1 + 1$	\newline
we can see $p(1)=1$, $p(2)=2$, $p(3)=3$, $p(4)=5$, $p(5)=7$
and \newline
$p(1,1)=1$,	\newline
$p(2,1)=1$, $p(2,2)=1$,	\newline
$p(3,1)=1$, $p(3,2)=1$, $p(3,3)=1$,	\newline
$p(4,1)=1$, $p(4,2)=2$, $p(4,3)=1$, $p(4,4)=1$,	\newline
$p(5,1)=1$, $p(5,2)=2$, $p(5,3)=2$, $p(5,4)=1$, $p(5,5)=1$,	\newline
from above. A partition of $n$ into $k$ parts has
two situations, one is each summand is greater than one,
in this situation, we subtract $1$ from each summand
and obtain a partion of $n-k$ into $k$ parts,
another situation is there is at least one $1$ in
the summand, in this situation, we eliminate a $1$
and obtain a partition of $(n-1)$ into $(k-1)$ parts.
Hence we got the induction formula
$$p(n,k)=p(n-1,k-1) + p(n-k,k)$$
therefore $p(n)=\sum_{k=0}^n p(n,k)$.

Observe the value of $p(n,k)$ for small $n$ and $k$
futher more, we found $p(n,1)=p(n,n)=1(n > 0)$, $p(n,2)=[{n \over 2}]$,
$p(n,n-1)=1(n > 1)$, $p(n,n-2)=2(n>3)$,
and $p(n,k)=P(n-1,k-1)$ when $2k>n$, becasue $p(n-k,k)=0$ when $n-k<k$.
Let $n=r+k+1$

\subsubsection{Stirling number of the second kind and Bell number}
Let $S(n,k)$ denote the number of ways to divide
a set of $n$ elements into $k$ non-empty disjoint
parts. Define $S(n,k)=0$ when $n < k$, $S(n,0)=0$ when $n>0$
and $S(0,0)=1$. Obviously $S(n,1)=S(n,n)=1$. \newline
Example:	\newline
all partitions of $\{1,2,3\}$ are:	\newline
$\{1,2,3\}$, \newline
$\{1\} \cup \{2,3\}$, $\{2\} \cup \{1,3\}$, $\{3\} \cup \{1,2\}$,	\newline
$\{1\} \cup \{2\} \cup \{3\}$,	\newline
therefore $S(3,2)=3$;
\newline
all partitions of of $\{1,2,3,4\}$ are: \newline
$\{1,2,3,4\}$,	\newline
$\{1\} \cup \{2,3,4\}$, $\{2\} \cup \{1,3,4\}$, $\{3\} \cup \{1,2,4\}$, $\{4\} \cup \{1,2,3\}$,	\newline
$\{1,2\} \cup \{3,4\}$, $\{1,3\} \cup \{2,4\}$, $\{1,4\} \cup \{2,3\}$,	\newline
$\{1\} \cup \{2\} \cup \{3,4\}$, $\{1\} \cup \{3\} \cup \{2,4\}$,	\newline
$\{1\} \cup \{4\} \cup \{2,3\}$, $\{2\} \cup \{3\} \cup \{1,4\}$,	\newline
$\{2\} \cup \{4\} \cup \{1,3\}$, $\{3\} \cup \{4\} \cup \{1,2\}$,	\newline
$\{1\} \cup \{2\} \cup \{3\} \cup \{4\}$,	\newline
therefore $S(4,2)=7$, $S(4,3)=6$.
We list the value of $S(n,k)$ for $1 \leq k \leq n \leq 4$ below:	\newline
$S(1,1)=1$,	\newline
$S(2,1)=1$, $S(2,2)=1$,	\newline
$S(3,1)=1$, $S(3,2)=3$,	$S(3,3)=1$,	\newline
$S(4,1)=1$, $S(4,2)=7$,	$S(4,3)=6$, $S(4,4)=1$.	\newline
Consider a partion of $\{1,2, \dots , n\}$ into $k$ parts,
$\{1,2, \dots , n\} = U_1 \cup U_2 \dots \cup U_k$.
If $\{n\}=U_i$ for some $i$, then the reset subsets
form a partition of $\{1,2, \dots , n-1 \}$ into $k-1$ parts,
otherwise, we eliminate $n$ form the subset it
belongs, that subset still remains non-empty, thus we got
a partition of $\{1,2, \dots , n-1\}$ into $k$ parts.
But given a partition of $\{1,2, \dots , n-1\}$ into $k$ parts,
we have $k$ ways to add $n$ into one of the $k$ subsets,
therefore we got the induction formula for $S(n,k)$, which is
$$S(n,k)=S(n-1,k-1) + k S(n-1,k).$$
The sum $B(n)=\sum_{k=0}^n S(n,k)$ is know as Bell number,
which denotes the number of all partitions of a set with
$n$ elements. $B(n)$ has a recursive formula
$$B(n+1)=\sum_{k=0}^n {n \choose k} B(k).$$

\section{Advanced Topics}
\subsection{Poset}
If there is a partial order $\leq$ on a set $P$, we say $(P,\leq)$, or simply $P$ is a poset.
A chain on a poset is a sequence $\{a_n\}$ such that $a_1\leq a_2\leq...\leq a_n$ and no two elements are equal,
$n$ is called the length of the chain.
An anti-chain is a subset $A$ of $P$ such that no two elements in that subset are compareble,
$\#A$ is called the length of the anti-chain.

The Dilworths' theorem and its dual theorem say:
the maximum length of a chain is equal to the minimum number of anti-chain cover;
the maximum length of an anti-chain is equal the mimimum number of chain cover.

\subsection{Incident Algebra}
\subsubsection{matrix and vector on poset}
A $n$ dimensional vector $v(v_1 , v_2 , \dots , v_n)$ on $K$
can be viewed as a function $$v : [n] \to K,$$
while a $n$ by $n$ matrix $[M_{ij}]$ on $K$ can be viewed as
a function $$M : [n] \times [n] \to K.$$ Now we consider
$[n]$ to be any another poset.

Let $P$ be a local bound poset,
and $\alpha, \beta$ be matrix, i.e.
$$\alpha , \beta : P \times P \to K,$$
define the product of $\alpha$ and $\beta$ by
$$(\alpha \beta)(x , y) = \sum_{x \leq z \leq y} \alpha(x,z) \beta(z,y).$$
If we restrict the entry $\alpha(x,y)$ in the matrix
to be non-zero only when $x \leq y$, the the product
we've defined agrees the product of two matrix.
Let $f, g$ be vectors, i.e.
$$f , g : P \to K,$$
define the product of vector and matrix by
$$(\alpha f)(x) = \sum_{x \leq y} \alpha(x,y)f(y)$$
and
$$(f \alpha)(x) = \sum_{y \leq x} f(y)\alpha(y,x)$$
vector can be either row vector or column vecotr,
depend on the matrix multiply it is on the left or right.
Let $$I : P \times P \to K, I(x,y) = \begin{cases} 1 & x \leq y \cr 0 & x \not\leq y \end{cases}$$
then
$$f(x) = \sum_{y \leq x} g(y)$$
can be write in the form
$$f = gI,$$
while
$$f(x) = \sum_{x \leq y} g(y)$$
can be write in the form
$$f = Ig.$$

\subsubsection{M\"obius inversion formula}
Let $\mu$ be the inverse of $I$ such that
$$\mu I = I \mu = \delta,$$
where
$$\delta(x , y) = \begin{cases} 1 & x = y\cr 0 & x \neq y \end{cases}.$$
then $$\delta(x,y) = \sum_{x \leq z \leq y} \mu(x,z) I(z,y)
= \sum_{x \leq z \leq y} \mu(x,z)$$
therefore $$\mu(x,x)=1$$
and $$\sum_{x \leq z \leq y}\mu(x,z) = 0, x \neq y$$
Also we have
$$\delta(x,y) = \sum_{x \leq z \leq y} I(x,z) \mu(z,y)
= \sum_{x \leq z \leq y} \mu(z,y),$$
and $$\sum_{x \leq z \leq y}\mu(z,y) = 0, x \neq y$$

Now when we know $f=gI$ and $f=Ih$,
we can calculate $g$ and $h$ as
$$g(x) = (f \mu)(x) = \sum_{y \leq x} f(y) \mu(y, x)$$
and
$$h(x) = (\mu f)(x) = \sum_{x \leq y} \mu(x, y) f(y).$$


\section{Constructive Combinatorics}

\section{String}
\subsection{The KMP Algorithm}
\subsection{Trie and Aho-Corasick Algorithm}
\subsection{Introdution to Automata and Language}
